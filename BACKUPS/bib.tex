\begin{thebibliography}{99}

\bibitem{urbanistyka}
	Chmielewski J.M.,	\emph{Teoria urbanistyki w projektowaniu i planowaniu miast},	Oficyna Wydawnicza Politechniki Warszawskiej, Warszawa 2001.

\bibitem{projektowanie_obiektow_motoryzacyjnych}
	Mikoś-Rytel W., Biedrońska J., Figaszewski J., Kozak K., Lisik A.,	\emph{Projektowanie obiektów motoryzacyjnych},	Wydawnictwo Politechniki Śląskiej, Gliwice 2008.

\bibitem{poznan}
	Zespół Blue Ocean Business consulting ds. transportu publicznego,
	\emph{Koncepcja budowy funkcjonalnych węzłów przesiadkowych PKM w kierunku zwiększenia ich dostępności oraz oferowania usług komplementarnych do komunikacji publicznej},
	Lipiec 2015.
	
\bibitem{metodyka}
	Olszewski P., Krukowska H., Krukowski P.,	\emph{Metodyka oceny wskaźnikowej węzłów przesiadkowych transportu publicznego},
	 ,,Transport miejski i regionalny'', 	czerwiec 2014.
	
\bibitem{metodyka2}
	Olszewski P., Krukowska H., Krukowski P., \emph{Jak oceniać projektowane lub funkcjonujące węzły przesiadkowe?}, ,,Biuletyn Komunikacji Miejskiej'' nr 143, 2017.
	
\bibitem{stanko}
	Stańko K.,	\emph{Przegląd i charakterystyka systemów parkingowych}, czasopismo ,,Mechanika'', 2012.
	
\bibitem{spillar}
	Spillar R.J.,	\emph{Park-and-ride planning and design guidelines},	Parsons Brinckerhoff Inc. 1997.
	
\bibitem{szarata}
	Szarata A.	\emph{Ocena efektywności funkcjonalnej parkingów przesiadkowych (P+R)},	praca doktorska, Politechnika Krakowska, październik 2005.

\bibitem{oxford}
		Oxford Mail, \emph{How Oxford led the way to create Park and Rides},	6 grudnia 2013,
	\url{http://www.oxfordmail.co.uk/news/10859209.How_Oxford_led_the_way_to_create_Park_and_Rides/}
	
\bibitem{szarata2}
	Szarata A.,
	\emph{Analiza wielkości parkingów Park and Ride zlokalizowanych w obszarach metropolitarnych}, czasopismo ,,Budownictwo i architektura'', kwiecień 2014.
	
\bibitem{makarova}
	Makarova I., Shubenkova K., Gabsalikhova L.,	\emph{Analysis of the city transport system's development strategy design pronciples with account of risks and specific features of spaial development},	czasopismo ,,Transport Problems'', Kazan Federal University, 12/2017.
	
\bibitem{olsson}
	Lindström Olsson A.,	\emph{Factors that influence choice of travel mode in major urban areas. The attractiveness of Park \& Ride},
	Division of Transportation and Logistics, KTH Royal Institute of Technology 2003.
	
\bibitem{gothenburg}
	Goteborgs Stad,
	\emph{Fysisk planering för kollektivtrafik (Fizyczne planowanie transportu zbiorowego)},
	raport nr 1/2004.
	
\bibitem{florida}
	Florida Department of Transportation,	\emph{State Park-and-Ride guide}, czerwiec 2012
	
\bibitem{guide}
	Terzis G.,	\emph{GUIDE: group for urban interchanges development and evaluation},	marzec 2018
	
\bibitem{prawo-lewisa}
	Szymalski W.,	\emph{Prawo Lewisa-Mogridge’a w Warszawie - wprowadzenie}, styczeń 2012,
	\url{http://www.zm.org.pl/?a=lewis-mogridge-14-00_wprowadzenie}
	
\bibitem{tomtom}
	TomTom Traffic Index 2017.
	\url{https://www.tomtom.com/en_gb/trafficindex/}
	
\bibitem{oxford2}
	Oxfordshire county council:
	\url{https://www.oxfordshire.gov.uk/cms/public-site/park-and-ride}
	
\bibitem{rps}
	Stacey R.,
	\emph{The effectiveness and sustainability of Park and Ride},
	12 czerwca 2009
	
\bibitem{rybczynska}
	Rybczyńska M.
	\emph{System strategicznych parkingów "Park and Ride"},
	\url{http://www.transport.um.warszawa.pl/transport-publiczny/system-strategicznych-parking-w-parkuj-i-jed}
	
\bibitem{eurotest}
	RACC foundation, MoviNews:
	\emph{Park \& Ride; State of the Art in Europe}. ,,EuroTest'' nr 9 -- marzec/kwiecień 2009.
	
\bibitem{dijk}
	Dijk M., Montalvo C.,	\emph{Policy frames of Park-and-Ride in Europe},	,,Journal of Transport Geography'', nr 19/2011
	
\bibitem{icaen}
	Katalońska Grupa ds. Efektywności Energii. Kataloński Instytut Energii,
	\emph{White Book of the Sustainable Mobility in the early XXI century}.
	
\bibitem{standardy_szwajcarskie}
	Szymalski W.,\emph{Standardy szwajcarskie dla węzłów przesiadkowych}, czasopismo ,,Transport'', 14 marca 2018.
	
\bibitem{makuch2}
	Makuch J., \emph{Projektowanie przystanków tramwajowych dla bezpieczeństwa i wygody pasażerów}, X Konferencja Naukowo-Techniczna ,,Drogi Kolejowe '99'' Spała, 13-15 października 1999.
	
\bibitem{oleksiewicz}
	Oleksiewicz W., Żurawski S., \emph{Drogi szynowe, podstawy projektowania linii i węzłów tramwajowych}, Zakład Inżynierii Komunikacyjnej Politechniki Warszawskiej, Warszawa 2004.
	
\bibitem{czauderna}
	Czauderna T., \emph{Konstrukcje torów tramwajowych}, czasopismo ,,TTS Technika Transportu Szynowego'' nr 9/2004.
	
\bibitem{wytyczne_ulice}
	Generalna Dyrekcja Dróg Publicznych, \emph{Wytyczne projektowania ulic}, Instytut Badawczy Dróg i Mostów, Warszawa, 1992.
	
\bibitem{ustawa_transport}
	Ustawa z dnia 16 grudnia 2010 r. o publicznym transporcie zbiorowym: Dz.U. 2011 nr 5 poz. 13.
	
\bibitem{rozporzadzenie_drogi}
	Rozporządzenie Ministra Transportu i Gospodarki Morskiej z dnia 2 marca 1999 r. 
	w sprawie warunków technicznych, jakim powinny odpowiadać drogi publiczne i ich usytuowanie (Dz. U. z 1999r. nr 43, poz. 430).
	
\bibitem{warunki_techniczne}
	Rozporządzenie Ministra Infrastruktury z dnia 12 kwietnia 2002 r. 
	w sprawie warunków technicznych, jakim powinny odpowiadać budynki i ich usytuowanie (Dz. U. z 2015r. poz. 1422)
	
\bibitem{knd_podatne}
	Generalna Dyrekcja Dróg Krajowych i Autostrad, \emph{Katalog typowych konstrukcji nawierzchni podatnych i półsztywnych}, opracowanie Katedry Inżynierii Drogowej Politechniki Gdańskiej, Gdańsk 2014.
	
\bibitem{knd_sztywne}
	Generalna Dyrekcja Dróg Krajowych i Autostrad, \emph{Katalog typowych konstrukcji nawierzchni sztywnych}, opracowanie Katedry Dróg i Lotnisk Politechniki Wrocławskiej, Wrocław 2014.
	
\bibitem{tp1}
	Czubiński R., \emph{Jak zaprojektować dobry węzeł przesiadkowy}, ,,Transport Publiczny'', 16 października 2010.
	
\bibitem{zaluski}
	Załuski D., \emph{Zintegrowane węzły przesiadkowe przy małych dworcach kolejowych}, 
	,,TTS Technika Transportu Szynowego'' str. 62--68, nr 21, 2014.
	
\bibitem{kazimierczyk}
	Kazimierczyk M., \emph{Koncepcja współczesnego węzła przesiadkowego -- praktyczny przykład rozwiązania}, Biuletyn Komunikacji Miejskiej IGKM, nr 136, maj 2015.
	
\bibitem{young}
	Young A., \emph{Manual for Streets. London.}, Thomas Telford Publishing, 2007.
	
\bibitem{guidelines_washington}
	Washington Metropolitan Area Transit Authority, \emph{Guidelines for station site and access planning}, Sierpień 2005.
	
\bibitem{standardy_wroclaw}
	Bocheńska-Niemiec A., Cebrat K., Kusowska K., Romanik A., Tyrka Ł., Walter E., Wiszniowski J., \emph{Wrocławskie standardy kształtowania przestrzeni miejskich przyjaznych pieszym}, Gmina Wrocław, maj 2017.
	
\bibitem{standardy_chodnik}
	Józefowicz P., \emph{Katalog Standardów Nawierzchni Chodników dla Wrocławia}, Wrocław 2013.
	
\bibitem{raport-vip}
	Bielecki P., \emph{Kształtowanie przyjaznych dla pieszych węzłów przesiadkowych w mieście}, Warszawska Inicjatywa Piesza, Zielone Mazowsze, listopad 2011.
	
\bibitem{makuch}
	Makuch J., wykład w formie elektronicznej, kurs Koleje Miejskie, Politechnika Wrocławska, \url{http://www.zits.pwr.wroc.pl/makuch/kmm_W1.pdf}
	
\bibitem{tory_tramwajowe}
	Rychlewski J., Firlik B., Straszewski W., \emph{Wytyczne projektowania torów tramwajowych a obecnie używany tabor tramwajowy}, Archiwum Instytutu Inżynierii Lądowej nr 25, 2017. 
	
	%GDANSK
	
	\bibitem{portal_gdansk}
	Portal Miasta Gdańska, inwestycje miejskie, \url{http://www.gdansk.pl/inwestycje-miejskie/Gdanski-Projekt-Komunikacji-Miejskiej-etap-III-A,a,17762}
	
	\bibitem{gazeta_gdansk}
	Dobaczewski J., Kraska A., Zomkomski S., \emph{ŁOŚ, czyli nowy Węzeł Integracyjny komunikacji miejskiej: Łostowice – Świętokrzyska}, Gazeta Metropolitalnego Związku Komunikacyjnego Zatoki Gdańskiej ,,Przystanek Metropolitarny'' nr 8, lipiec 2013, \url{http://www.zkmgdynia.pl/admin/__pliki__/290x400_x8_mzkzg_PrzystanekNr8%20internet.pdf}
	
	\bibitem{opracowanie_gdansk}
	Mordak R., WYG International Sp. z o.o., \emph{Studium wykonalności dla zadań inwestycyjnych i modernizacyjnych przewidzianych do realizacji w latach 2008-2011}, Gdański Projekt Komunikacji Miejskiej etap IIIa, Warszawa, listopad 2008.
	
	\bibitem{maciej_lada}
	Łada M., Birr K. \emph{Analiza zmian funkcjonowania transportu zbiorowego wynikających z budowy węzłów integracyjnych}, III Krakowska Ogólnopolska Konferencja Naukowa Transportu ,,KOKONAT'' Kraków, 21-22 kwietnia 2016.
	
	\bibitem{kaszubowski}
	Kaszubowski D., \emph{Badanie jakości usług transportu zbiorowego na nowej trasie tramwajowej w dzielnicy Gdańsk Południe}, ,,Technika Transportu Szynowego'' nr 10/2013.
	
	%WROCLAW
	
	\bibitem{grunwaldzki1}
	Korycki T., Molecki B., Puchalski P., Wicher M., \emph{Historia i przebudowa węzła autobusowego przy placu Grunwaldzkim we Wrocławiu}, ,,Przewoźnicy i systemy transportowe'', nr 11/2008.
	
	\bibitem{grunwaldzki2}
	Gisterek I., wykład wykład w formie elektronicznej, Zakład Infrastruktury Transportu Szynowego, Politechnika Wrocławska, \url{http://www.zm.org.pl/download/prezentacje/0909-gisterek_pwroc.pdf}
	
	\bibitem{grunwaldzki3}
	Stowarzyszenie ,,Akcja miasto'', \emph{Raport o ruchu pieszych we Wrocławiu}, 2015.
	
	\bibitem{grunwaldzki4}
	Molecki B., \emph{Analiza ruchu pieszego w obrębie węzłów przesiadkowych na przykładzie placu Grunwaldzkiego we Wrocławiu}, Konferencja naukowo-techniczna ,,Zintegrowany system transportu miejskiego'', Wrocław, 27-28 maja 2010.
	
	%KRAKOW
	
	\bibitem{biezanow1}
	Gazeta Wyborcza Kraków, \emph{Weźże zaparkuj i jedź tramwajem. Park\& Ride już w Bieżanowie}, \url{http://krakow.wyborcza.pl/krakow/7,44425,22901021,wezze-zaparkuj-i-jedz-tramwajem-park-ride-juz-w-biezanowie.html}
	
	\bibitem{biezanow2}
	Miejska Infrastruktura Kraków, \emph{P\& R Bieżanów}, \url{http://mi.krakow.pl/parkingi/inwestycja-biezanow}
	
	\bibitem{biezanow3}
	Wiadomości nasze-miasto.pl Kraków, \emph{Parking Park\& Ride w Bieżanowie}, 2 lipca 2017, \url{http://krakow.naszemiasto.pl/artykul/parking-park-ride-w-biezanowie-wizualizacje,3440403,art,t,id,tm.html}
	
	\bibitem{biezanow4}
	Magiczny Kraków, \emph{Parking P+R w Nowym Bieżanowie otwarty}, 15 stycznia 2018, \url{http://krakow.pl/aktualnosci/216797,1912,komunikat,parking_p+r_w_nowym_biezanowie_otwarty.html}
	
	%WARSZAWA
	
	\bibitem{mlociny1}
	Pudło J., {Węzeł Metro Młociny -- lokalizacja i znaczenie}, InfoBus, 14 lipca 2013, \url{http://www.infobus.pl/wezel-metro-mlociny-lokalizacja-i-znaczenie_more_44933.html}
	
	\bibitem{mlociny2}
	Urbanowicz W., \emph{Warszawa: Jak poprawić węzeł? ZTM ,,ulepszy'' Młociny}, 14 marca 2016, Transport Publiczny, \url{http://www.transport-publiczny.pl/wiadomosci/jak-poprawic-wezel-ztm-ulepszy-mlociny-51577.html}
	
	\bibitem{mlociny3}
	Oficjalny serwis Metra Warszawskiego, \emph{Budowa tunelu B23 i stacji A23 Młociny wraz z węzłem komunikacyjnym}, \url{http://www.metro.waw.pl/budowa-tunelu-b23-i-stacji-a23-mlociny-wraz-z-wezlem-komunikacyjnym}
	
	\bibitem{mlociny4}
	Jackowski M., \emph{Węzeł komunikacyjny Młociny okiem pasażera}, II Warsztaty Forum LINK w Bydgoszczy, 21 września 2009.
	
	
	
	\bibitem{oltaszyn}
	Gołębiewska M., \emph{Trasa tramwajowa na Ołtaszyn będzie dłuższa? Jest szansa, że tramwaj obsłuży też Wysoką}, Serwis TuWrocław,\url{https://www.tuwroclaw.com/wiadomosci,trasa-tramwajowa-na-oltaszyn-bedzie-dluzsza-jest-szansa-ze-tramwaj-obsluzy-tez-wysoka,wia5-3273-38956.html}
	
	
\end{thebibliography}
	