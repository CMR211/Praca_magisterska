
%PDFLatex obsolete packages:
%	\usepackage[utf8]{inputenc}
%	\usepackage[T1]{fontenc}
%	\usepackage[polish]{babel}
%	\usepackage{lmodern}

%LuaTex packages:
	\usepackage{fontspec}
	\setmainfont{Times New Roman}\let\emph\textit
	\usepackage{polski}
	\defaultfontfeatures{Ligatures=TeX} %ligatury w luatex

%tekst:
	\usepackage{setspace} \onehalfspacing %interlinie 1.5
	%\usepackage{indentfirst} %aby pierwszy paragraf również był wcięty
	%\usepackage[ampersand]{easylist} %łatwiejsze listy
	\usepackage{verbatim} %środowisko kodu
	\usepackage{microtype} %poprawki tekstu
	%\usepackage{titlesec} %tytuły sekcji i rozdziałów
	\usepackage{textcomp} %symbole text companion

%tabele:
	\usepackage{hhline} %linie poziome w tabelach
	\usepackage{tabu}
	\usepackage{booktabs}
	\usepackage{bigstrut}
	\usepackage{multirow}

%nagłówki:
	\usepackage{fancyhdr} 

%matematyka
	\usepackage{amsmath} %zapis matematyczny
	\usepackage{amsfonts} %znaki matematyczne
	\usepackage{siunitx} %jednostki SI
	\usepackage{fp} %obliczenia zmiennoprzecinkowe
	\usepackage{gensymb} %symbole
	\usepackage{ifthen}

%figury:
	\usepackage{float} %aby użyć [H]
	\usepackage{pgfplots}	\pgfplotsset{width=10cm,compat=1.9} %obrazki z pdfów
	\usepackage{caption} \captionsetup{justification=centering,labelfont=bf} %podpisy pod obrazkami
	\usepackage{graphicx}	\graphicspath{ {img/} } %importowanie obrazów
	\usepackage{subcaption} %dodatkowe captions dla subfigur
	
%wdowy i sieroty
	%\widowpenalty=10000
	%\clubpenalty=10000

	\usepackage{hyperref} %odnosniki
	\hypersetup{colorlinks=true, linkcolor=black, filecolor=magenta, urlcolor=blue, citecolor=black}
