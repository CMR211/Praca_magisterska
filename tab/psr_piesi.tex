% Table generated by Excel2LaTeX from sheet 'Sheet2'
\begin{table}[H]
  \centering
  \caption{Poziomy swobody ruchu pieszych}
    \begin{tabular}{p{1cm}p{3cm}p{2cm}p{8cm}}
    \hline
    PSR & Pow. na 1 osob [m2/os] & Gęstość $k$ [os/m2] & Warunki ruchu \bigstrut[b]\\
    \hline
    A          & $\geq 5.5$  & $0 \div 0.1$ & Swoboda poruszania się, brak konieczności zmiany toru ruchu \bigstrut[t]\\
    B          & $3.7 \div 5.5$ & $0.1 \div 0.25$ & Sporadyczna konieczność zmiany toru ruchu \\
    C          & $2.2 \div 3.7$ & $0.24 \div 0.4$ & Częsta konieczność zmiany toru ruchu \\
    D          & $1.4 \div 2.2$ & $0.4 \div 0.7$ & Ograniczenie prędkości poruszania oraz możliwość wyprzedzania wolniejszych pieszych \\
    E          & $0.8 \div 1.4$ & $0.7 \div 1.8$ & Ograniczenie prędkości poruszania, przy bardzo ograniczonej możliwości wyprzedzania wolniejszych pieszych \\
    F          & $\leq 0.8$ & $\geq 1.8$ & Bardzo duże ograniczenie prędkości poruszania, częsty kontakt z innymi pieszymi \\
    \hline
    & & & \\
    \end{tabular}%
  \label{psr_piesi}
  
  \zrodlo{źródło: Bocheńska-Niemiec A., Cebrat K., Kusowska K., Romanik A., Tyrka Ł., Walter E., Wiszniowski J., \emph{Wrocławskie standardy kształtowania przestrzeni miejskich przyjaznych pieszym}, Gmina Wrocław, maj 2017 \cite{standardy_wroclaw}}
\end{table}%
